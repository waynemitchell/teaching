\documentclass[18pt,xcolor=table]{beamer}

%!TEX root = ./main.tex
\usepackage {bbm}
\usepackage {textpos}
\usepackage {tikz}
\usepackage {graphicx}


\definecolor{blue1}{RGB}{176,196,222 }
\definecolor{blue2}{RGB}{54,100,139}

\definecolor{grey1}{RGB}{139,139,131}
\definecolor{grey2}{RGB}{235,235,235}

\definecolor{black1}{RGB}{50,50,50}

\mode<presentation>
{
  % \usetheme{Pittsburgh}   
  %\usetheme{Boadilla}  
\usetheme{Madrid}  
  \usefonttheme[onlymath]{serif}
  \setbeamertemplate{items}[circle] 
  \setbeamertemplate{sections/subsections in toc}[circle]
  \setbeamercovered{invisible}
  \setbeamertemplate{navigation symbols}{}
% \usecolortheme{seahorse}
%
%  % Color Theme 
  \setbeamercolor{normal text}{bg=white,fg=black1} %All standard text
  \setbeamercolor{structure}{fg=blue2} %% Table of Contents 

  \setbeamercolor*{frametitle}{fg=black1,bg=grey2} % Frame title colors
%  \setbeamerfont{frametitle}{series=\bfseries}
  \setbeamercolor*{framesubtitle}{fg=blue2} % Frame subtitle color

  \setbeamercolor*{palette primary}{use=structure,fg=black1, bg=grey2} %right bottom
  \setbeamercolor*{palette secondary}{use=structure,bg=blue1} %middle bottom
  \setbeamercolor*{palette tertiary}{use=structure,bg=blue2,fg=grey2} %left bottom

  \setbeamercolor*{block body}{fg=black1,bg=blue1!10} % Color of blocks
  \setbeamercolor*{block title}{parent=structure,fg=black1,bg=blue1} % Block Titles
  \setbeamercolor{alerted text}{fg=blue2!85!black,} % Alerted Text (ie. highlight with \alert)
  \setbeamerfont{alerted text}{series=\bfseries}

  % not sure what these do.
  \setbeamercolor{item projected}{use=item,fg=black1,bg=item.fg!35}
  \setbeamercolor*{block title alerted}{parent=alerted text,bg=black1!15}
  \setbeamercolor*{block title example}{parent=example text,bg=black1!15}
  \setbeamerfont{framesubtitle}{size=\small}
}

\makeatletter
\setbeamertemplate{footline}
{
  \leavevmode%
    \hbox{%
      \begin{beamercolorbox}[wd=.333333\paperwidth,ht=2.25ex,dp=1ex,center]{author in head/foot}%
        \usebeamerfont{author in head/foot}\insertshortauthor%~~\beamer@ifempty{\insertshortinstitute}{}{(\insertshortinstitute)}
      \end{beamercolorbox}%
        \begin{beamercolorbox}[wd=.333333\paperwidth,ht=2.25ex,dp=1ex,center]{title in head/foot}%
        \usebeamerfont{title in head/foot}\insertshorttitle
        \end{beamercolorbox}%
        \begin{beamercolorbox}[wd=.333333\paperwidth,ht=2.25ex,dp=1ex,right]{date in head/foot}%
        \usebeamerfont{date in head/foot}\insertshortdate{}\hspace*{2em}
        \insertframenumber{} / \inserttotalframenumber\hspace*{2ex} 
      \end{beamercolorbox}}%
        \vskip0pt%
}
\makeatother

%\usepackage{kerkis}
\usepackage{helvet} 
\usepackage[T1]{fontenc}
\usepackage[protrusion=true,expansion=true]{microtype}
\usepackage{amsmath}

\renewcommand*{\thefootnote}{\fnsymbol{footnote}}


\pgfdeclareimage[height=1.5cm]{logo}{./logos/hd_logo}
\pgfdeclareimage[height=0.8cm]{small_logo}{./logos/hd_logo}

\AtBeginSection[] { 
  \begin{frame}[plain] 
    \frametitle{\bf Outline:}
    \framesubtitle{~~} 
    \tableofcontents[currentsection] 
  \end{frame} 
  \addtocounter{framenumber}{-1} 
} 

\setbeamercovered{transparent}

%%%%%%%%%%%%%%%%%%%%%%%
% user-defined commands
%%%%%%%%%%%%%%%%%%%%%%%
%!TEX root = ../main.tex


\newcommand{\beq}{\begin{equation}}
\newcommand{\eeq}{\end{equation}}

\newcommand{\eq}[1]{\begin{align*}#1\end{align*}}

\newcommand{\bfi}{\begin{figure}}
\newcommand{\efi}{\end{figure}}

\newcommand{\icg}{\includegraphics}

\newcommand{\bdm}{\begin{displaymath}}
\newcommand{\edm}{\end{displaymath}}

\newcommand{\beqa}{\begin{eqnarray}}
\newcommand{\eeqa}{\end{eqnarray}}

\newcommand{\beqas}{\begin{eqnarray*}}
\newcommand{\eeqas}{\end{eqnarray*}}

\newcommand{\barr}{\begin{array}}
\newcommand{\earr}{\end{array}}

\newcommand{\bit}{\begin{itemize}}
\newcommand{\eit}{\end{itemize}}

\newcommand{\qq}[1]{\qquad \mbox{#1} \qquad}

\def\hyph{-\penalty0\hskip0pt\relax}

% Tikz
\definecolor{blue1}{RGB}{176,196,222 }
\definecolor{blue2}{RGB}{54,100,139}
\definecolor{grey1}{RGB}{139,139,131}
\definecolor{grey2}{RGB}{235,235,235}
\definecolor{black1}{RGB}{50,50,50}


%Spaces
\newcommand{\Sp}[1]{{\cal #1}}
%
\newcommand{\sA}{\Sp{A}}
\newcommand{\sB}{\Sp{B}}
\newcommand{\sC}{\Sp{C}}
\newcommand{\sD}{\Sp{D}}
\newcommand{\sE}{\Sp{E}}
\newcommand{\sF}{\Sp{F}}
\newcommand{\sG}{\Sp{G}}
\newcommand{\sH}{\Sp{H}}
\newcommand{\sI}{\Sp{I}}
\newcommand{\sJ}{\Sp{J}}
\newcommand{\sK}{\Sp{K}}
\newcommand{\sL}{\Sp{L}}
\newcommand{\sM}{\Sp{M}}
\newcommand{\sN}{\Sp{N}}
\newcommand{\sO}{\Sp{O}}
\newcommand{\sP}{\Sp{P}}
\newcommand{\sQ}{\Sp{Q}}
\newcommand{\sR}{\Sp{R}}
\newcommand{\sS}{\Sp{S}}
\newcommand{\sT}{\Sp{T}}
\newcommand{\sU}{\Sp{U}}
\newcommand{\sV}{\Sp{V}}
\newcommand{\sW}{\Sp{W}}
\newcommand{\sX}{\Sp{X}}
\newcommand{\sY}{\Sp{Y}}
\newcommand{\sZ}{\Sp{Z}}

%Vectors
\newcommand{\V}[1]{{\bf #1}}
%
\newcommand{\va}{\V{a}}
\newcommand{\vb}{\V{b}}
\newcommand{\vc}{\V{c}}
\newcommand{\vd}{\V{d}}
\newcommand{\ve}{\V{e}}
\newcommand{\vf}{\V{f}}
\newcommand{\vg}{\V{g}}
\newcommand{\vh}{\V{h}}
\newcommand{\vi}{\V{i}}
\newcommand{\vj}{\V{j}}
\newcommand{\vk}{\V{k}}
\newcommand{\vl}{\V{l}}
\newcommand{\vm}{\V{m}}
\newcommand{\vn}{\V{n}}
\newcommand{\vo}{\V{o}}
\newcommand{\vp}{\V{p}}
\newcommand{\vq}{\V{q}}
\newcommand{\vr}{\V{r}}
\newcommand{\vs}{\V{s}}
\newcommand{\vt}{\V{t}}
\newcommand{\vu}{\V{u}}
\newcommand{\vv}{\V{v}}
\newcommand{\vw}{\V{w}}
\newcommand{\vx}{\V{x}}
\newcommand{\vy}{\V{y}}
\newcommand{\vz}{\V{z}}

\newcommand{\vA}{\V{A}}
\newcommand{\vB}{\V{B}}
\newcommand{\vC}{\V{C}}
\newcommand{\vD}{\V{D}}
\newcommand{\vE}{\V{E}}
\newcommand{\vF}{\V{F}}
\newcommand{\vG}{\V{G}}
\newcommand{\vH}{\V{H}}
\newcommand{\vI}{\V{I}}
\newcommand{\vJ}{\V{J}}
\newcommand{\vK}{\V{K}}
\newcommand{\vL}{\V{L}}
\newcommand{\vM}{\V{M}}
\newcommand{\vN}{\V{N}}
\newcommand{\vO}{\V{O}}
\newcommand{\vP}{\V{P}}
\newcommand{\vQ}{\V{Q}}
\newcommand{\vR}{\V{R}}
\newcommand{\vS}{\V{S}}
\newcommand{\vT}{\V{T}}
\newcommand{\vU}{\V{U}}
\newcommand{\vV}{\V{V}}
\newcommand{\vW}{\V{W}}
\newcommand{\vX}{\V{X}}
\newcommand{\vY}{\V{Y}}
\newcommand{\vZ}{\V{Z}}

\newcommand{\vone}{\V{1}}
\newcommand{\vzero}{\V{0}}
\newcommand{\B}{\V{B}}
\newcommand{\E}{\V{E}}
\newcommand{\Er}{\V{E}_r}
\newcommand{\Es}{\V{E}_s}
\newcommand{\un}{\hat{\vn}}



%Vectors
\newcommand{\T}[1]{\underline{\bf #1}}
%
\newcommand{\ta}{\T{a}}
\newcommand{\tb}{\T{b}}
\newcommand{\tc}{\T{c}}
%\newcommand{\td}{\T{d}}
\newcommand{\te}{\T{e}}
\newcommand{\tf}{\T{f}}
\newcommand{\tg}{\T{g}}
%\newcommand{\th}{\T{h}}
\newcommand{\ti}{\T{i}}
\newcommand{\tj}{\T{j}}
\newcommand{\tk}{\T{k}}
\newcommand{\tl}{\T{l}}
\newcommand{\tm}{\T{m}}
\newcommand{\tn}{\T{n}}
%\newcommand{\to}{\T{o}}
\newcommand{\tp}{\T{p}}
\newcommand{\tq}{\T{q}}
\newcommand{\tr}{\T{r}}
\newcommand{\ts}{\T{s}}
%\newcommand{\tt}{\T{t}}
\newcommand{\tu}{\T{u}}
\newcommand{\tv}{\T{v}}
\newcommand{\tw}{\T{w}}
\newcommand{\tx}{\T{x}}
\newcommand{\ty}{\T{y}}
\newcommand{\tz}{\T{z}}

\newcommand{\tA}{\T{A}}
\newcommand{\tB}{\T{B}}
\newcommand{\tC}{\T{C}}
\newcommand{\tD}{\T{D}}
\newcommand{\tE}{\T{E}}
\newcommand{\tF}{\T{F}}
\newcommand{\tG}{\T{G}}
\newcommand{\tH}{\T{H}}
\newcommand{\tI}{\T{I}}
\newcommand{\tJ}{\T{J}}
\newcommand{\tK}{\T{K}}
\newcommand{\tL}{\T{L}}
\newcommand{\tM}{\T{M}}
\newcommand{\tN}{\T{N}}
\newcommand{\tO}{\T{O}}
\newcommand{\tP}{\T{P}}
\newcommand{\tQ}{\T{Q}}
\newcommand{\tR}{\T{R}}
\newcommand{\tS}{\T{S}}
\newcommand{\tT}{\T{T}}
\newcommand{\tU}{\T{U}}
\newcommand{\tV}{\T{V}}
\newcommand{\tW}{\T{W}}
\newcommand{\tX}{\T{X}}
\newcommand{\tY}{\T{Y}}
\newcommand{\tZ}{\T{Z}}

\newcommand{\tone}{\T{1}}
\newcommand{\tzero}{\T{0}}

%Matrix
\newcommand{\M}[1]{{\mathbb #1}}
%
\newcommand{\mA}{\M{A}}
\newcommand{\mB}{\M{B}}
\newcommand{\mC}{\M{C}}
\newcommand{\mD}{\M{D}}
\newcommand{\mE}{\M{E}}
\newcommand{\mF}{\M{F}}
\newcommand{\mG}{\M{G}}
\newcommand{\mH}{\M{H}}
\newcommand{\mI}{\M{I}}
\newcommand{\mJ}{\M{J}}
\newcommand{\mK}{\M{K}}
\newcommand{\mL}{\M{L}}
\newcommand{\mM}{\M{M}}
\newcommand{\mN}{\M{N}}
\newcommand{\mO}{\M{O}}
\newcommand{\mP}{\M{P}}
\newcommand{\mQ}{\M{Q}}
\newcommand{\mR}{\M{R}}
\newcommand{\mS}{\M{S}}
\newcommand{\mT}{\M{T}}
\newcommand{\mU}{\M{U}}
\newcommand{\mV}{\M{V}}
\newcommand{\mW}{\M{W}}
\newcommand{\mX}{\M{X}}
\newcommand{\mY}{\M{Y}}
\newcommand{\mZ}{\M{Z}}

\newcommand{\mzero}{\M{0}}
\newcommand{\mone}{\M{1}}

% Derivatives
\newcommand{\pd}[2]{\frac{\partial #1}{\partial #2}}
\newcommand{\ppd}[2]{\frac{\partial^2 #1}{\partial #2^2}}
\newcommand{\td}[2]{\frac{\mathrm{d} #1}{\mathrm{d} #2}}

\newcommand{\px}{ \partial_{x} }
\newcommand{\py}{ \partial_{y} }
\newcommand{\pz}{ \partial_{z} }
\newcommand{\pt}{ \partial_{t} }
\newcommand{\ptt}{ \partial_{tt} }

\newcommand{\Div}[1]{\nabla \cdot #1}
\newcommand{\Curl}[1]{\nabla \times #1}
\newcommand{\Ctwo}[1]{\nabla_2 \times #1}
\newcommand{\Grad}[1]{\nabla #1}
\newcommand{\Gperp}[1]{\nabla^\perp #1}
\newcommand{\Lap}[1]{\Delta #1}

%integral d
\newcommand{\dd}[0]{\, \mathrm{d}}

% common discrete quantities
\newcommand{\dt}[0]{\delta t}

% physical variables
\newcommand{\eps}[0]{\epsilon_0}
\newcommand{\mus}[0]{\mu_0}
\newcommand{\boltz}[0]{\kappa_B}
\newcommand{\s}[0]{\alpha}
\newcommand{\vths}[0]{v_{th_\s}}
\newcommand{\vthe}[0]{v_{th_e}}
\newcommand{\vthi}[0]{v_{th_i}}
\newcommand{\Om}[0]{\Omega}
\newcommand{\bdOm}{\partial \Omega}

% bracketing
\newcommand{\inner}[2]{\langle #1, #2 \rangle}
\newcommand{\lb}[0]{\left[}
\newcommand{\rb}[0]{\right]}
\newcommand{\parn}[1]{\left( #1 \right)}
\newcommand{\la}{\langle}
\newcommand{\ra}{\rangle}
\newcommand{\lcb}{\left\{}
\newcommand{\rcb}{\right\}}

\newcommand{\mathAnd}{\,\,\mbox{and}\,\,}
\newcommand{\mathOn}{\,\,\mbox{on}\,\,}

\newcommand{\h}{\hat}
\newcommand{\wh}{\widehat}
%\newcommand{\ul}{\underline}

% math operators
\DeclareMathOperator{\Trace}{trace}
\DeclareMathOperator{\Supp}{supp}
\DeclareMathOperator{\Span}{span}
\DeclareMathOperator{\floor}{floor}
\DeclareMathOperator{\diam}{diam}
\DeclareMathOperator{\ceil}{ceil}
\DeclareMathOperator*{\argmin}{arg\,min}


\newcommand{\red}[1]{\textcolor{red}{#1}}
%added macro definitions here

\usepackage{tikz}
\usepackage{tabularx}
\usetikzlibrary{decorations.markings}
\usetikzlibrary{arrows,positioning} 

\usepackage{cancel}
\usepackage{hyperref}
\usepackage{caption}
\usepackage{subcaption}
\usepackage[]{algorithm}
\usepackage{algpseudocode}
\captionsetup{compatibility=false}


\title[Multigrid]{Introduction to Multigrid Methods}
\subtitle{Day 4: Advanced Problems}
\author[Mitchell]{Wayne Mitchell}
\institute{\pgfuseimage{logo}\\Universit\"at Heidelberg\\Institut f\"ur Technische Informatik}
\date[]{\alert{}}


\begin{document}
%!TEX root = ./main.tex
\tikzstyle{block} = [rectangle, draw, rounded corners, shade, top color=white, text width=5em,
  bottom color=blue!50!black!20, draw=blue!40!black!60, very thick, text centered, minimum height=4em]
  \tikzstyle{line} = [draw, -latex']
  \tikzstyle{cloud} = [draw, ellipse,top color=white, bottom color=red!20, node distance=2cm, minimum height=2em]

  \frame{\titlepage}

  \addtobeamertemplate{frametitle}{}{%
      \begin{textblock*}{100mm}(0.9\textwidth,-0.88cm)
    \pgfuseimage{small_logo}
    \end{textblock*}
  }

\AtBeginSection[] { 
  \begin{frame}[t]
    \frametitle{\bf Outline:}
    \framesubtitle{~~} 
    \tableofcontents[currentsection] 
  \end{frame} 
  \addtocounter{framenumber}{-1} 
} 

\let\tempone\itemize
\let\temptwo\enditemize
\renewenvironment{itemize}{\tempone\addtolength{\itemsep}{0.5\baselineskip}}{\temptwo}

\DeclareRobustCommand{\Chi}{\raisebox{2pt}{$\chi$}}
%%%%%%%%%%%%%%%%%%%%%%%%%%%%%%%%%%%%%%%%%%%%%%%%%%%%%%%%%%%%%%%%%%%%%%%%%%%%%%%%

% Slide
\begin{frame}
\frametitle{\bf Outline:}
\framesubtitle{~~}
\tableofcontents
\end{frame}

%%%%%%%%%%%%%%%%%%%%%%%%%%%%%%%%%%%%%%%%%%%%%%%%%%%%%%%%%%%%%%%%%%%%%%%%%%%%%%%%

\section{Introduction}

% Slide
\begin{frame}{Introduction}
\begin{block}{Advanced problems}
\bit
\item Previous lectures focused on multigrid for symmetric positive definite (SPD) problems resulting from discretization of elliptic PDEs
\item Today's lecture examines a few extensions to more advanced problems:
\bit
\item Non-linear
\item Non-symmetric
\item Non-PDE
\eit
\eit
\end{block}
\end{frame}

%%%%%%%%%%%%%%%%%%%%%%%%%%%%%%%%%%%%%%%%%%%%%%%%%%%%%%%%%%%%%%%%%%%%%%%%%%%%%%%%

\section{Non-linear problems}

% TODO model problem?

% Slide
\begin{frame}{Non-linear problems}
\begin{block}{Definition of linearity}
\bit
\item $A$ is linear if:
\bit
\item $A(\mathbf{u} + \mathbf{v}) = A(\mathbf{u}) + A(\mathbf{v})$, if $\mathbf{u},\mathbf{v}$ arbitrary vectors
\item $A(\alpha\mathbf{u}) = \alpha A(\mathbf{u})$, if $\alpha$ is a scalar
\eit
\eit
\end{block}
\end{frame}

% Slide
\begin{frame}{Non-linear problems}
\begin{block}{Definition of linearity}
\bit
\item Consider $A = -\Delta$. Then,
\bit
\item $A(u + v) = -\Delta(u + v) = -\Delta u - \Delta v$
\item $A(\alpha u) -\Delta (\alpha u) = -\alpha \Delta u$
\eit
\item Now consider $A(u) = u^2$
\bit
\item $A(u + v) = (u + v)^2 \neq u^2 + v^2$
\item $A(\alpha u) = (\alpha u)^2 \neq \alpha u^2$
\eit
\eit
\end{block}
\end{frame}

% Slide
\begin{frame}{Non-linear problems}
\begin{block}{Newton's method}
\bit
\item Newton's method is an iterative solution method for non-linear equations
\item Consider a non-linear equation in 1D, $A(\tilde{u}) = 0$
\item Given initial guess, $u$, and error, $e = \tilde{u} - u$, consider the Taylor expansion about $u$:
\eq{
   0 = A(\tilde{u}) = A(u + e) = A(u) + eA'(u) + \frac{s^2}{2}A''(\xi)
}
for $\xi$ between $u$ and $\tilde{u}$
\eit
\end{block}
\end{frame}

% Slide
\begin{frame}{Non-linear problems}
\begin{block}{Newton's method}
\bit
\item Neglecting higher-order terms, solve for the correction, $e$:
\eq{
   0 &= A(u) + eA'(u) \\
   e &= -\frac{A(u)}{A'(u)}
}
\item Use approximate error to iteratively update the guess
\eq{
   u \leftarrow u - \frac{A(u)}{A'(u)}
}
\item The above iteration is Newton's method
\eit
\end{block}
\end{frame}

% Slide
\begin{frame}{Non-linear problems}
\begin{block}{Newton's method}
\bit
\item Consider the $N$-dimensional non-linear system:
\eq{
   A(\mathbf{\tilde{u}}) 
   = 
   \begin{bmatrix}
   A_1(\tilde{u}_1,\tilde{u}_2,...,\tilde{u}_N) \\
   A_2(\tilde{u}_1,\tilde{u}_2,...,\tilde{u}_N) \\
   \vdots \\
   A_N(\tilde{u}_1,\tilde{u}_2,...,\tilde{u}_N)
   \end{bmatrix} 
   = 
   \begin{bmatrix}
   0 \\
   0 \\
   \vdots \\
   0
   \end{bmatrix} 
   =
   \mathbf{0}
}
\eit
\end{block}
\end{frame}

% Slide
\begin{frame}{Non-linear problems}
\begin{block}{Newton's method}
\bit
\item Taylor expansion in $N$ dimensions about guess, $\mathbf{u}$, with error, $\mathbf{e}$:
\eq{
   \mathbf{0} = A(\mathbf{u} + \mathbf{e}) = A(\mathbf{u}) + J(\mathbf{u})\mathbf{e} + \text{higher-order terms}
}
where $J(\mathbf{u})$ is the Jacobian matrix $J$ evaluated at $\mathbf{u}$
\eq{
   J(\mathbf{u})
   = 
   \begin{bmatrix}
   \frac{\partial A_1(\mathbf{u})}{\partial u_1} & \frac{\partial A_1(\mathbf{u})}{\partial u_2} & \dots & \frac{\partial A_1(\mathbf{u})}{\partial u_N} \\
   \frac{\partial A_2(\mathbf{u})}{\partial u_1} & \frac{\partial A_2(\mathbf{u})}{\partial u_2} & \dots & \frac{\partial A_2(\mathbf{u})}{\partial u_N} \\
   \vdots \\
   \frac{\partial A_N(\mathbf{u})}{\partial u_1} & \frac{\partial A_N(\mathbf{u})}{\partial u_2} & \dots & \frac{\partial A_N(\mathbf{u})}{\partial u_N}
   \end{bmatrix}    
}
\eit
\end{block}
\end{frame}

% Slide
\begin{frame}{Non-linear problems}
\begin{block}{Newton's method}
\bit
\item Newton's method for systems:
\eq{
   u \leftarrow u - (J(u))^{-1}A(u)
}
\item Note: the non-linear problem has been reduced to iteratively solving a sequence of linear systems!
\eit
\end{block}
\end{frame}

% Slide
\begin{frame}{Non-linear problems}
\begin{block}{Newton-Multigrid}
\bit
\item Obvious application of multigrid: solve $(J(u))^{-1}A(u)$ with multigrid
\item This approach is called Newton-Multigrid and can work well
\item We would like to apply multigrid principles directly to the non-linear problem
\eit
\end{block}
\end{frame}

% Slide
\begin{frame}{Non-linear problems}
\begin{block}{Non-linear relaxation}
\bit
\item Need a relaxation method for the non-linear system, $A(\mathbf{u}) = \mathbf{f}$
\item Non-linear Gauss-Seidel: 
\bit
\item For $j = 1...N$:
\item Solve the $j^{th}$ equation for the $j^{th}$ component of the solution vector:
\eq{
   A_j(\mathbf{u}+s\mathbf{\epsilon}_j) = f_j
}
where $\mathbf{\epsilon}_j$ is the $j^{th}$ canonical basis vector
\item Can use scalar Newton's method for the solve above
\item Update solution vector: $\mathbf{u} \leftarrow \mathbf{u} + s\mathbf{\epsilon}_j$
\eit
\eit
\end{block}
\end{frame}

% Slide
\begin{frame}{Non-linear problems}
\begin{block}{Non-linear coarse-grid correction}
\bit
\item Coarse-grid correction is based on solving the residual equation
\item In the linear case, $A\mathbf{e} = A(\mathbf{\tilde{u}} - \mathbf{u}) = \mathbf{f} - A\mathbf{u}$
\item Not so in the non-linear case!
\eit
\end{block}
\end{frame}

% Slide
\begin{frame}{Non-linear problems}
\begin{block}{Non-linear coarse-grid correction}
\bit
\item Still want to solve the residual equation for a correction, $\mathbf{e}$:
\eq{
   \mathbf{f} - A(\mathbf{u}) = \mathbf{r} \\
   A(\mathbf{u} + \mathbf{e}) - A(\mathbf{u}) = \mathbf{r}
}
\item Furthermore, want to solve for the correction on a coarse grid:
\eq{
   A^c(\mathbf{u}^c + \mathbf{e}^c) - A^c(\mathbf{u})^c = \mathbf{r}^c 
}
\eit
\end{block}
\end{frame}

% Slide
\begin{frame}{Non-linear problems}
\begin{block}{Non-linear coarse-grid correction}
\bit
\item Can restrict the residual $\mathbf{r}^c$ from the fine grid as usual:
\eq{
   \mathbf{r}^c = R\mathbf{r} = R(\mathbf{f} - A(\mathbf{u}))
}
\item Similarly, restrict the current solution iterate, $\mathbf{u}$:
\eq{
   \mathbf{u}^c = R\mathbf{u}
}
\eit
\end{block}
\end{frame}


% Slide
\begin{frame}{Non-linear problems}
\begin{block}{Non-linear coarse-grid correction}
\bit
\item Substitute these into the coarse-grid residual equation:
\eq{
   A^c(R\mathbf{u} + \mathbf{e}^c) - A^c(R\mathbf{u}) = R\mathbf{r} \\
   A^c(\mathbf{\tilde{u}^c}) = A^c(R\mathbf{u}) + R\mathbf{r} 
}
\item Coarse-grid correction:
\bit
\item Solve for $\mathbf{\tilde{u}^c}$: 
\eq{
   A^c(\mathbf{\tilde{u}^c}) = A^c(R\mathbf{u}) + R(\mathbf{f} - A(\mathbf{u})) 
}
\item Get the coarse-grid correction:
\eq{
   \mathbf{e}^c = \mathbf{\tilde{u}}^c - R\mathbf{u}
}
\item Interpolate and add the correction:
\eq{
   \mathbf{u} \leftarrow \mathbf{u} + P\mathbf{e}^c
}
\eit
\eit
\end{block}
\end{frame}

% Slide
\begin{frame}{Non-linear problems}
\begin{block}{Interpolation, restriction, and coarse-grid operator}
\bit
\item Often the same linear interpolation and full-weighting restriction used in the linear case are sufficient (can also use higher-order geometric interpolation)
\item Coarse-grid operator is usually obtained through rediscretization on the coarse grid
\item Can also sometimes form $A^c(\mathbf{u}^c) = RA(\mathbf{u})P$
\eit 
\end{block}
\end{frame}

% Slide
\begin{frame}{Non-linear problems}
\begin{block}{Full approximation scheme (FAS)}
\bit
\item Now have all the pieces for a multigrid cycle
\begin{algorithm}[H]
\caption{Full approximation scheme (FAS) two-grid cycle}
\begin{algorithmic}
\State Set $\mathbf{u}$ initial guess
\State Do non-linear relaxation on $A(\mathbf{u}) = \mathbf{f}$
\State Calculate residual $\mathbf{r} = \mathbf{f} - A(\mathbf{u})$
\State Restrict residual $\mathbf{r}^c = R\mathbf{r}$
\State Restrict approximate solution $\mathbf{u}^c = R\mathbf{u}$
\State Solve on the coarse grid $A^c(\mathbf{\tilde{u}}^c) = A^c(R\mathbf{u}) + R\mathbf{r}$
\State Get the coarse-grid correction $\mathbf{e}^c = \mathbf{\tilde{u}^c} - \mathbf{u}^c$
\State Interpolate coarse-grid correction $\mathbf{u} = \mathbf{u} + P\mathbf{e}^c$
\State Do non-linear relaxation on $A(\mathbf{u}) = \mathbf{f}$
\end{algorithmic}
\end{algorithm}
\eit
\end{block}
\end{frame}


% Slide
\begin{frame}{Non-linear problems}
\begin{block}{Full approximation scheme (FAS)}
\bit
\item Similar extension to multilevel as in the linear case: V-cycles, W-cycles, FMG
\item Note that if $A$ is linear, FAS reduces to regular multigrid cycling
\item FAS is a fixed point iteration 
\eit
\end{block}
\end{frame}

% Slide
\begin{frame}{Non-linear problems}
\begin{block}{$\tau$ correction}
\bit
\item 
\eit
\end{block}
\end{frame}

% Slide
\begin{frame}{Non-linear problems}
\begin{block}{FMG and the basin of attraction}
\bit
\item 
\eit
\end{block}
\end{frame}




% Slide
\begin{frame}{Non-linear problems}
\begin{block}{}
\bit
\item 
\eit
\end{block}
\end{frame}

% FMG FAS and basin of attraction

%%%%%%%%%%%%%%%%%%%%%%%%%%%%%%%%%%%%%%%%%%%%%%%%%%%%%%%%%%%%%%%%%%%%%%%%%%%%%%%%

\section{Non-symmetric problems}

% Slide
\begin{frame}{Non-symmetric problems}
\begin{block}{}
\bit
\item 
\eit
\end{block}
\end{frame}

% Slide
\begin{frame}{Non-symmetric problems}
\begin{block}{}
\bit
\item 
\eit
\end{block}
\end{frame}


%%%%%%%%%%%%%%%%%%%%%%%%%%%%%%%%%%%%%%%%%%%%%%%%%%%%%%%%%%%%%%%%%%%%%%%%%%%%%%%%

\section{Graph problems}

% Slide
\begin{frame}{Graph problems}
\begin{block}{}
\bit
\item 
\eit
\end{block}
\end{frame}

% Slide
\begin{frame}{Graph problems}
\begin{block}{}
\bit
\item 
\eit
\end{block}
\end{frame}


%%%%%%%%%%%%%%%%%%%%%%%%%%%%%%%%%%%%%%%%%%%%%%%%%%%%%%%%%%%%%%%%%%%%%%%%%%%%%%%%

\end{document}

